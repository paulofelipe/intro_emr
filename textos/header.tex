%títulos da tabela em cima
\usepackage{floatrow}
\floatsetup[figure]{capposition=top}
\floatsetup[table]{capposition=top}

% Estilos do caption
\usepackage{caption}
\captionsetup[figure]{labelsep=period}
\captionsetup[table]{labelsep=period}

% cores dos links e ajustamento horizontal
\usepackage{hyperref}
\hypersetup{pdfstartview=FitH,colorlinks=true,linkcolor=black,urlcolor=blue,citecolor=blue,bookmarksopen,bookmarksdepth=3}

%cria o ambiente landscape
\usepackage{pdflscape}
\newcommand{\blandscape}{\begin{landscape}}
\newcommand{\elandscape}{\end{landscape}}

%identação do parágrafo
\setlength{\parindent}{0.7cm} % Default is 15pt.

% estilo das seções
\usepackage{titlesec}
\titleformat{\section}[block]
{\normalfont\bfseries\centering}{\thesection.}{0.5em}{\MakeUppercase}{}

\titleformat{\subsection}[block]
{\normalfont\bfseries}{\thesubsection}{0.5em}{}

\titleformat{\subsubsection}[block]
{\normalfont\bfseries}{\thesubsubsection}{0.5em}{}

\titleformat{\subsubsubsection}[block]
{\normalfont\bfseries}{\thesubsubsubsection}{0.5em}{\newline}

% Pacote para cores
\usepackage[dvipsnames]{xcolor}

% estilo do header e do footer
\usepackage{fancyhdr}
\pagestyle{fancy}
\renewcommand{\sectionmark}[1]{\markright{\tiny #1}} 
\fancyhf{}
\makeatletter
  \fancyhead[RE]{\footnotesize{\sffamily\textcolor{darkgray}{\@title}}~~\normalsize{\thepage}}
\makeatother
\makeatletter
  \fancyhead[LO]{\normalsize{\thepage}~~\footnotesize{\sffamily\textcolor{darkgray}{\@title}}}
\makeatother
\rfoot{}
\renewcommand{\headrulewidth}{0pt}

% espaçemento da linha
\renewcommand{\baselinestretch}{1.1}

% altera a fonte
\usepackage[T1]{fontenc}
\usepackage{fbb}

% Alterando estilo do título
\makeatletter
\def\@maketitle{%
  \newpage
  \thispagestyle{empty}
  \null
  \vskip 2em%
  \begin{center}%
  \let \footnote \thanks
  {\Large \bfseries \MakeUppercase \@title \par}%
    \vskip 1.5em%
  \begin{flushright}
  %\vskip 10em%
    {\large
      \lineskip .2em%
      \sffamily \@author
      \par}%
  \end{flushright}
    \vskip 1em%
    %{\large \@date}%
  \end{center}%
  \par
  \vskip 1.5em}
\makeatother

% Muda estilo no sumário
%\usepackage{tocloft}
%\renewcommand{\cftsecfont}{\sffamily\bfseries}
%\renewcommand{\cftsubsecfont}{\sffamily}
%\renewcommand{\cftsubsubsecfont}{\sffamily}

% Ícones
\usepackage{fontawesome}

% Informações do título
\setlength{\droptitle}{-2em}

  \title{Equilíbrio Geral Computável: Implementando o MINIMAL no R}
    \pretitle{\vspace{\droptitle}\centering\huge}
  \posttitle{\par}
    \author{Paulo Felipe Oliveira\\
\textnormal{\sffamily \faGithub~paulofelipe} \\
\textnormal{\sffamily\faTwitter~paulofelipeao}}
    \preauthor{\centering\large\emph}
  \postauthor{\par}
      \predate{\centering\large\emph}
  \postdate{\par}
    \date{05/06/2019}

% tamanho da fonte dos códigos

\let\oldShaded\Shaded
\let\endoldShaded\endShaded
\renewenvironment{Shaded}{\footnotesize\oldShaded}{\endoldShaded}

%% change fontsize of output
\let\oldverbatim\verbatim
\let\endoldverbatim\endverbatim
\renewenvironment{verbatim}{\footnotesize\oldverbatim}{\endoldverbatim}